
\documentclass[12pt]{article}
\usepackage{fancyhdr}     % Enhanced control over headers and footers 
\usepackage[T1]{fontenc}  % Font encoding
\usepackage{mathptmx}     % Choose Times font 
\usepackage{microtype}    % Improves line breaks      
\usepackage{setspace}     % Makes the document look like horse manure 
\usepackage{lipsum}       % For dummy text
\usepackage[utf8]{inputenc}

\usepackage{csquotes}

\pagestyle{fancy} % Default page style 
\lhead{Aaron Craig, PHIL442}
\chead{}
\rhead{\thepage}
\lfoot{}
\cfoot{}
\rfoot{}

\thispagestyle{empty} %First page style 

\setlength\headheight{15pt} %Slight increase to header size

\begin{document}
\begin{center}
\begin{tabular}{c}
\large Evaluating Carnap's \textit{Semantics and Ontology} \\
Aaron Craig, \today. \\
PHIL442 (Max Cresswell) \\



\end{tabular}
\end{center}
%\doublespacing

\noindent
If we accept empiricism, what becomes of ontology? Can we remain empirical while speaking about entities such as propositions and theorems? These are the questions addressed in Carnap's \textit{Empiricism, Semantics, and Ontology}. Carnap's goal is to allow the empiricist to speak about the semantics of abstract objects without committing to their existence or reducing them to meaningless formalisms. He takes the example of mathematics and physics:

\begin{displayquote}
``In the case of mathematics some empiricists try to find a way out by treating the whole of mathematics as a mere calculus... the mathematician is said to speak not about numbers, functions, and infinite classes but merely about meaningless symbols and formulas.... In physics it is more difficult to shun the suspected entities because the language of physics serves for the communication of reports and predictions and hence cannot be taken as a mere calculus.'' (205)
\end{displayquote}

What's under consideration is the ontological status of these entities. In particular, we are interested in existence questions, such as ``are there numbers?'', or ``are there laws of physics?'' On Carnap's account, existentials are of two sorts: internal questions are asked within the rules and customs of a particular linguistic framework, and external questions judge the worth of using a particular framework. With this internal/external distinction, and an adequate notion of ``framework'', the empiricist may speak freely of propositions and theorems, for ``such a language does not imply embracing a Platonic ontology...'' (206) This idea generalises to let one talk about any abstract objects without having to believe our speech denotes something real.

This essay addresses the notion of a framework and whether the internal/external distinction successfully resolves the problem of talking about abstract objects as an empiricist. We begin with a summary of Carnap's paper before more closely scrutinising frameworks and whether they can meaningfully relate to factual matters.

\section{The Internal-External Distinction}

To talk about a particular kind of entity requires three things: (213)

\begin{enumerate}
	\item A new term to name the kind.
	\item Variables, which stand for entities of that kind.
	\item An explanation of the rules governing how we are to speak about these entities.
\end{enumerate}

These three things make up a framework. The first two allow us to name the kind itself, and to write down sentences which refer to entities of that kind. The third requires an explanation of what are the entities of that kind, their characteristics, how they may be used and distinguished, and so on. Carnap demonstrates with propositions. These, he explains, are declarative sentences, where ``p is a proposition'' is defined by ``p'' or ``not p''. For example, ``Chicago is a city'' is a proposition, because either ``Chicago is a city'' or ``Chicago is not a city''\footnote{~We are only interested in what kind of entity an utterance represents; therefore, ``not Chicago is a city'' has been rewritten as the more natural sounding ``Chicago is not a city''.}. Another framework is that of numbers. Carnap gives an especially important, non-analytic framework called the \textit{thing world}, which gives us a way to interpret sense-data. For example, you may confirm or deny ``the ink on this page is black'' by looking at the ink on this page and experiencing what it is like to see black.

Internal questions are about the existence of particular entities inside a framework. For example, we can ask ``are there numbers?'' in the framework of numbers. This has a trivial answer: yes! One is such a number, as is two, and three, and four. If we were operating in a framework of propositions that had no conception of numbers, then the answer is again trivial: no. In the thing framework, an internal question would assert the existence of a physical object in space-time: ``To recognise something as a real thing or event means to succeed in incorporating it into the system of things at a particular space-time position so that it fits with the other things as real, according to the rules of the framework.'' (207)

On the other hand, an external question is about the worth of operating in a framework. Answering an external question requires us to assess the framework's value towards a particular ends. For example, we might ask if it is worth talking about numbers. A positive answer might justify the framework of numbers by its practical use in counting and interpreting quantities. For a more substantive example, consider the engineers and physicists who make GPS devices. At the scales and distances involved, time dilation has a significant effect on the result of the GPS, so using quantum mechanics, which accounts for time dilation, produces a more notably accurate result than using Newtonian mechanics, which does not account for time dilation. By judging these two systems in terms of which gives the more accurate results, we may respond to the question ``What framework of mechanics is the best?'' with ``quantum''.

To talk about entities now is to engage in a particular way of speaking, governed by the framework describing their kind. This only requires the speaker to accept the new framework; they do not have to assert the existence of entities of that kind. We frame our discourse in terms of frameworks and the meanings they give us, rather than believing our words denote real, abstract objects. On this account, a mathematician be an empiricist and talk freely about mathematical objects and their meaning without having to believe in them as real, abstract entities.

Carnap does not say much about physics, or whether his account elevates it to a level above mere calculus. \textit{Prima facie}, there may seem to be little difference between physics as a framework and physics as a calculus --- but there is one. As a calculus, physics is like a symbolic game involving the derivation of certain quantities and results according to the application of physical and mathematical laws. These laws are purely analytic and do not speak to any physical reality. Utterances in this calculus have no factual connection. It is a frozen, fixed system; neither Newtonian nor quantum mechanics ever changes. We may need factual observations to discover what are their rules, but once they have been established, they do not change. By comparison, physics as a framework is an organic enterprise. It may contain several calculi, but these are employed according to a set of rules, such as the scientific method, which determine whether we should accept new calculi, what calculi we should apply to a certain state of affairs, and when we should apply them. Physics as a framework is therefore more than a mindless calculus, for it equips us with the ability to develop new understandings of when to apply particular calculi in response to factual matters.

The framework of physics highlights a problem in applying the analytic/synthetic distinction to frameworks, as Carnap does when he refers to frameworks being ``logical or analytic'' (205). The framework of physics does not neatly fit into either category, because it has both analytic statements (``if propelled through a vacuum an object will stay in motion'') and factual statements (``that particular object is in motion'')\footnote{~The first is analytic because it follows from the definition of ``propelled'' and Newton's first law. The second is synthetic because it requires us to confirm a particular object's motion through time-space.}. So too does the framework of things: an inference about sense-experience requires us to mix analytic and synthetic, as in this example:

\begin{enumerate}
	\item The ink on this page is black.
	\item This page is part of Aaron's essay.
	\item Therefore, part of Aaron's essay has black ink.
\end{enumerate}

While establishing the validity of the argument is an analytic activity, establishing 1. and 2. is a factual one. Establishing the soundness of the argument requires both. Fortunately, there is no problem in abolishing the analytic/synthetic distinction for frameworks; it is not necessary for Carnap's argument, but it is at least a mistake.

We have now established that, on Carnap's account, talking about abstract objects such as mathematics and physics does not require us to believe in their existence; it only requires adopting a particular manner of speaking. If that is so, how can we make sense of ontological questions about them? For Carnap, if a question like ``is there X'' to have cognitive significance, it must be interpreted as either an internal or external question. If it is internal, it can be answered by applying the rules of the framework under consideration. For example, if someone asked ``are there numbers?'' then, operating inside the framework of numbers, we may give a positive answer by citing numbers such as one, two, or three, which live in this framework. A more complicated existential, such as ``is there a greatest prime number?'' may require a more careful, sophisticated application of rules, but can be shown to have a negative answer by applying the rules and definitions of number theory. In the framework of things, we can answer ``is there black ink on this page?'' by using the rules of the thing language to associate the scribbles on the page with a particular colourful experience in space-time. The other way to interpret an ontological question about an abstract entity is to cast it as an external question. This is, as Carnap mentions, a value judgement/

One might object to this characterisation of existentials by asking ``but are there \textit{really} numbers?'' As Carnap explains, this objection seems to be wanting some ontological justification prior to the acceptance of the framework we use to talk about numbers. As we have established, to accept numbers is to adopt a particular manner of speaking about their kind. As Carnap puts it: ``the introduction of the new ways of speaking does not need any theoretical justification because it does not imply any assertion of reality'' (214). To ask if some entity exists, prior and external to any particular way of speaking about it, is therefore cognitively meaningless.

At this point we have summarised Carnap's ideas and how they might be applied to solve the issues facing an empirical treatment of the languages of mathematics and physics. We now turn our attention towards more troubling aspects of his project.

\section{Defining Frameworks}

Carnap is vague about what a framework \textit{is}, exactly. In particular, it is not clear what constitutes a sufficient description for some kind of entity. Take his framework of propositions, where ``Chicago is a city is a proposition'' means ``Chicago is a city'' or ``Chicago is not a city''. How is this to be understood? Evidently, ``Stop!'' is not a proposition, while ``Chicago is a city'' is. But what is it about Carnap's meaning that witnesses one as a valid proposition, but not the other? To witness something as a proposition requires us only to apply the definition of what is a proposition in the framework; therefore, to answer this question, we should analyse Carnap's explanation of what ``P is a proposition'' means. To do that, we shall consider different, refined meanings of ``P is a proposition'', taking ``Chicago is a city'' as our example.

``P is a proposition'' might be valid if either ``P'' or ``not P'' is a syntactically valid statement in some language. Then ``Chicago is a city'' is grammatical English, but neither ``Stop!'' nor ``Don't stop!'' are grammatical statements, so we can acknowledge ``Chicago is a city is a proposition'', but not ``Stop! is a proposition.'' However, adopting a syntactic meaning leads to unintended consequences, because such meanings fail to separate a proposition from its surface utterances. Consider ``Chicago is a city'' and ``Ko t\a=etahi t\a=aone Chicago''. These are different utterances of the same proposition, but if the meaning of ``P is a proposition'' is given by the syntactic structure of P, then we are forced to say they are different. This is surely not what it means to be a proposition. 

Another attempted solution is to posit that ``Chicago is a city'' and ``Ko t\a=etahi t\a=aone Chicago'' are propositions because of what we can verify of their factual contents. Since these two utterances reference the same factual matters, and can be verified in principle by the same observations, they are the same proposition. One issue with this approach is the vagueness of most predicates relating to factual matters. For instance, what is a city? If we defined a city as a large population centre, we are prone to a Sorites' paradox: at which person did the population centre become large? We might adopt a meaning of ``X is a city'' which can be certainly and easily verified, such as ``Chicago is a legal city according to American law''. Then to determine if ``Chicago is a city'', we need only look at the referent of ``Chicago'' to see if it makes sense to apply the meaning of ``is a city'' or ``is not a city''. But even before civil planning, ``Ur is a city'' was a meaningful proposition in some slightly different framework of cities. These all seem like valid ways to determine if something is a city, but settling on one framework to tell us what ``X is a city'' means will produce a slightly different framework telling us what ``P is a proposition'' means, and makes propositions \textit{a posteriori} to cities, and indeed to any matter of fact that could be referenced by P. This is not necessarily a problem, but for even a conceptually simple framework, such as that of propositions, we are depending on a chain of other frameworks. Where does this chain end? Since our meaning is informed by the factual contents of an utterance, it must end at the framework governing our experience of sense-data; the framework of things. This solution is fine, provided we are willing to accept that every way of talking is \textit{a posteriori} to our experience of sense-data, which an empiricist may well accept. This seems to resolve the issue of what a framework is, because we can understand frameworks by considering how they relate to the fundamental framework by which we experience sense-data. But as we'll see in the next section, the idea of a fundamental framework governing our experience of factual matters is inherently non-empirical.

Lastly, we might eliminate all ambiguity from ``P is a proposition'' by adopting some meanings as axiomatic and using those as a basis for a formal semantics of everything else. This escapes the black-hole of ambiguity, but only places purely analytic frameworks on solid footing. In pursuing well-defined frameworks, they become the mere calculi Carnap sought to escape.

\section{Factual Frameworks}

From our examination of the meaning of propositions, we established that any framework whose meaning is informed by factual matters must depend on the way we experience sense-data. This is itself governed by a framework which equips us with a way of parsing sense-data as sense-experiences. We shall call any such framework an experience framework. Understanding experience frameworks is crucial, for it is only through them that other frameworks may connect with factual matters. This is important, for it is what elevates them above the purely analytic, ``mere calculi'', which Carnap seeks to avoid. As Carnap points out though, there is no real choice in what experience framework we use:

\begin{displayquote}
``There is usually no deliberate choice because we all have accepted the thing language early in our lives as a matter of course. Nevertheless, we may regard it as a matter of decision in this sense: we are free to choose to continue using the thing language or not; in the latter case we could restrict ourselves to a language of sense data and other ``phenomenal'' entities, or construct an alternative to the customary thing language with another structure, or, finally, we could refrain from speaking.'' (207)
\end{displayquote}

The thing framework parses sense-data as events in space-time. ``This evaluation is usually carried out... as a matter of habit rather than a deliberate, rational procedure. But it is possible, in a rational reconstruction, to lay down explicit rules for the evaluation.'' (207) Carnap never elaborates on this bold claim, but if pressed to explain what rules legitimise a thing-statement like ``the ink on this page is black'', a response might be to say that we knew this particular colourful sensation based on the frequency at which the ink was emitting photons. But even before the science of colour, we still understood what ``the ink on this page is black'' meant and had an adequate understanding, not only of how to verify such a statement, but what it would mean factually for such a statement to be true (or false). What isn't clear is how this can be explained as a way of speaking, as the thing framework attempts. It is also untenable to suggest that we experience colour in a fundamentally different framework from our pre-scientific ancestors. Either way, Carnap's explanation for what the thing framework \textit{is}, is sorely lacking.

Recall Carnap's explanation of  the thing language: ``To recognise something as a real thing or event means to succeed in incorporating it into the system of things at a particular space-time position so that it fits with the other things as real, according to the rules of the framework.'' (207) Exercising the thing language is not a matter of acknowledging something's existence. Rather, it is a manner of speaking about a particular way in which a piece of sense-data is parsed into a sense-experience. This is an important difference between, say, mathematics as a framework, and mathematics as referring to existent abstract objects; the former implies a kind of choice about how we speak. It implies that there is not only one way to talk about things \textit{in principle}, but many, and that these are all equally meaningful ways of talking about things in their own frameworks. What is special about the thing framework is that it is the only experience framework we adopt \textit{in practice}. It could be that our inability to liberate ourselves from the thing framework is an incidental consequence of the way things are, but it is empirically meaningless to deny the thing framework. To see why, let us consider what its alternatives might look like \textit{in principle}.

We could have a plurality of frameworks, each equally significant. Could we say that one of these frameworks is more ``faithful'' to the world than the others? This appears to be an external question about whether we should accept a particular experience framework on the grounds of it being the most ``faithful'' of all. Instead of talking about a particular ends --- faithfulness --- let us instead refer to any general ends P. If a pragmatic question about a framework towards the ends P were meaningful, there should be some means of measuring, observing, or comparing, by which we establish one experience framework as better than another for the sake of P. But to make such a measurement requires us to already be inside some experience framework, so asking an external question about an experience framework only makes sense if it is relativised to some other experience framework. External questions about experience frameworks are therefore internal questions asked inside some other experience framework. This means it is not possible to establish, on any grounds, according to any pragmatic reason, whether it is better to adopt one way of talking about how we experience reality than another, prior to already adopting some way of talking about reality for non-pragmatic reasons. In other words, we commit ourselves to several modes of being, all equally plausible, none holding more sway than the others --- we are now ontological pluralists. It is troubling for an empiricist to affirm this position, because now how we experience factual matters depends on how we talk about them. 

Carnap says we might refrain from speaking. It is not entirely clear what this would entail, but if we stopped literally speaking, our other faculties would still process sense-data in the usual way. To truly ``stop speaking'' then would require us to divorce ourselves of any experience of physical reality --- but this spontaneous disconnect seems impossible. If we talk about an object that cannot be impacted by sense-data and cannot impact sense-data, we are no longer talking about a physical object. It is not empirically sound to talk about a physical, sensing object as disconnecting from reality because it adopted a new way of talking --- if such a thing is coherent, it is impossible.

Our only remaining option is to adopt the thing framework as the canonical picture of reality, because it embodies some quality of faithfulness which exists prior to all other experience frameworks. If this quality is analytical, then we are only accepting the thing framework as the canonical picture of reality because we have defined it to be so. But this quality cannot be factual, because it exists prior to any experience framework with which it might be perceived, and being empiricists, we require existent object sto be capable of being experienced, even if only in principle. To continue with this argument seems to require us to believe then in the existence of an abstract quality --- faithfulness --- which is prior to frameworks, and which the thing framework embodies. This is non-empirical, so we cannot do that.

We are ready to present the main argument against Carnap's account of ontology. We could not, even \textit{in principle}, imagine what it would mean to deny the thing framework and experience the world according to the rules of some other framework, so it is meaningless to talk about any sense-experience, except as we understand it in the thing framework. To accept the thing framework requires either a commitment to ontological pluralism; a faith in it as embodying some \textit{a priori} abstract quality; or as a wholly meaningless lingusitic framework we accept for non-pragmatic reasons. The first two are non-empirical, while the third makes the way in which we experience the world an arbitrary, uninformed choice. More than that, it is a choice which cannot be informed, because any pragmatic decision is \textit{a posteriori} to some acceptance of an experience framework. Therefore, no pragmatic reason can be given to accept the only way of experiencing reality --- a self-defeating outcome for the empiricist, and a reason to reject Carnap's ontology.

\section{Conclusions}

Carnap sought to frame ontological questions in purely empirical terms, and equip the empiricist with a means of talking about mathematical, scientific, and other seemingly abstract entities without committing to their existence. Insofar as his frameworks are analytic, they succeed at this task. Where they attempt to relate to factual matters, they fail. Most problematic is the thing framework, a suspiciously universally-accepted framework that allows us to incorporate real things and events into space-time. By following the consequences of Carnap's account, we have established that the only meaningful ontological questions about how we experience sense-data are either true by definition, or are relativised to some other experience framework. Therefore, Carnap's account gives us no way of accepting the framework for any pragmatic reason, unless we either become ontological pluralists or accept the thing framework as embodying some \textit{a priori} abstract quality (both non-empirical). 

Based on this, we must conclude that Carnap's frameworks do not adequately solve the problems of speaking about mathematics, physics, and other abstract entities while remaining empirical. Despite their use as formal tools, they cannot be taken as an exhaustive characterisation of ontology.\\

\noindent
\textbf{Word Count: 3800}


\section*{Bibliography}

\noindent
Carnap, R. (1950). `Empiricism, Semantics, and Ontology', in \textit{Revue Internationale de Philosophie} 4: 20-40. Reprinted in the Supplement to \textit{Meaning and Necessity: A Study in Semantics and Modal Logic}, enlarged edition (University of Chicago Press, 1956).

\end{document}











