
\documentclass[12pt]{article}
\usepackage{fancyhdr}     % Enhanced control over headers and footers 
\usepackage[T1]{fontenc}  % Font encoding
\usepackage{mathptmx}     % Choose Times font 
\usepackage{microtype}    % Improves line breaks      
\usepackage{setspace}     % Makes the document look like horse manure 
\usepackage{lipsum}       % For dummy text
\usepackage[utf8]{inputenc}
\usepackage{amssymb}

\usepackage{csquotes}

\pagestyle{fancy} % Default page style 
\lhead{Aaron Craig, PHIL442}
\chead{}
\rhead{\thepage}
\lfoot{}
\cfoot{}
\rfoot{}

\thispagestyle{empty} %First page style 

\setlength\headheight{15pt} %Slight increase to header size

\begin{document}
\begin{center}
\begin{tabular}{c}
\large Universal Prescriptivism \\
Aaron Craig, \today. \\
PHIL442 (Ed Mares) \\



\end{tabular}
\end{center}
%\doublespacing

\noindent
Rather than ascribing descriptive properties such as ``good'' or ``just'' to particular actions, prescriptivists interpret moral judgements as issuing universal rules about how to behave in principle. We will take a close look at R.M. Hare's version of the theory --- universal prescriptivism --- and whether it can be taken as a rational foundation for understanding ethics and morals. 

An early theory quite like prescriptivism can be found in the work of Rudolf Carnap. Carnap analysed a moral instruction like ``you ought to do X'' as an imperative, ``do X''. For example, ``killing is wrong'' means ``do not kill''. Because imperatives are not true or false, he concluded that moral judgements have no logical content: ``...we cannot deduce any propositions about future experiences. Thus this statement is not verifiable and has no theoretical sense, and the same thing is true of all other value statements''\footnote{~Carnap, 1937: 7.}. Carnap's theory is simple, but lacking in many ways. In examining these deficiencies, we shall highlight some important challenges that will be used to test Hare's prescriptivism.

We often feel that moral judgements should be objective. For instance, it cannot be the case that ``you ought to kill'' and ``you ought not to kill'' for different people in the same situation. But on Carnap's theory, one could shout ``life'' and another ``death'' and, since imperatives are not logical entities, there would be no contradiction in doing this. In reducing moral judgements to imperatives, Carnap stripped them of all their normative power. To him, this was probably not an issue; any attachment we feel to a particular moral judgement would be akin to a subjective preference: ``these are psychological, not philosophical assertions''\footnote{~ibid.}. Carnap believes we are tricked into treating moral judgements as asserting real, descriptive truths because they are given in the indicative mood, i.e. as a statement, rather than in their true imperative form.

We also feel that moral language should be universal. Carnap's account is unclear as to the scope of the corresponding imperatives. In saying ``killing is wrong'', are we saying to the listener they should not kill in this one situation? Does it mean they should never kill from the utterance onwards? Sometimes the context matters: killing might be wrong in peacetime, but in self-defence or justified wars, we might well call it right. Does the utterance ``killing is wrong'' apply in those situations? It is also sensible to morally evaluate past events, as when we say things like ``we shouldn't have conducted the dawn raids''. Such statements break Carnap's theory, because they correspond to an imperative about something in the past. How could we instruct someone to do something in the past?

The most interesting part of Carnap's theory is that it seeks an account of moral judgements by examining how they are used. In reducing them to imperatives, he appeals to that aspect of moral language which compels us to perform actions which are morally desirable. He also noted the apparent indiscrepancy between the superficial form of a moral judgement --- as an indicative --- and the behaviour it invites us to follow --- an imperative. Both of these are important to Hare, who takes the same starting points in \textit{The Language of Morals}. After examining his version of prescriptivism, we shall look to see if it is an ethical theory which is objective and universal. It is ultimately against these goals that it shall be measured.\\

\section{Universal Prescriptivism}

Hare assumes the most important function of moral language is to prescribe behaviours, and that it can be done in a logical way. His first task is therefore to show that imperatives are logical. Carnap thought they weren't logical, because a sentence like ``do the dishes'' doesn't really have a truth-value. But as in this paradox by Jørgen Jørgensen, imperatives can be used in seemingly valid inferences\footnote{~Jørgensen, 1938: 290.}:\\

(1) Keep your promises.\\
\indent
(2) This is a promise of yours.\\
\indent
$\therefore$ Keep this promise.\\

This example, and others like it, are sufficient to convince Hare that imperatives are subject to their own sort of logic. But other than giving a few rough guidelines\footnote{~Hare, 1952: 28}, he says very little about what his imperative logic looks like, or how we should interpret its premises and inferences. This is a weakness we shall revisit in greater depth.

After justifying the possibility of an imperative logic, Hare begins to explain moral language by unpacking the ways in which it is used. His conclusion is this: in finding ourselves confronted with the question ``what ought I do?'' a person can advocate a particular way of acting on principle; when choosing which course of action to take, we consider the effects of each, and ``...to choose effects \textit{because} they are such and such is to begin to act on a principle that such and such effects are to be chosen.''\footnote{~Hare, 1952: 59.}. Principles govern every action. To justify this, Hare points to how we can, when told ``X is good'' or ``you ought to X'', ask why, and be given reasons. Sometimes when asked why, we may not be able to clearly state the reasons, ``...but this does not mean our actions do not proceed according to principles''\footnote{~ibid: 64.}.

Now, when someone issues a moral judgement, they are advocating a particular rule or guideline about how to act in principle under particular circumstances: ``they refer to, and express acceptance of, a standard which has an application to other similar instances.''\footnote{~ibid: 129.}. Moral words are for moral instruction, and moral instructions prescribe how we should act in particular circumstances, so in assenting to a moral judgement we are committing ourselves to a universal rule --- a sort of imperative --- that applies in all like circumstances. This use of words, to promulgate certain standards about how to act, is prescriptive. That our prescriptions apply to every person in like circumstances is what makes them universal, hence the name ``universal prescriptivism''.

Many ethical systems work by treating a sentence ``X is good'' as descriptive, in that it associates some naturalistic or transcendent quality ``good'', which is prior to or above physical existence, to the action X. Hare's theory is not like this. In saying ``X is good'', we are prescribing X as desirable ends, and when we say ``you ought to X'', we are prescribing a particular course of action X in like circumstances. It is unclear in \textit{The Language of Morals} as to what kind of truths are moral truths, but he clarifies in a later work, \textit{Freedom and Reason}, that moral principles are synthetic\footnote{~Hare, 1963: 23.}. They are universal only in the logical sense that by assenting to them, we assent to the entailed imperative --- the universal rule --- which binds us to act in a particular way in like circumstances\footnote{~ibid: 30-1.}; they are universal in a \textit{logical} sense, not in a \textit{metaphysical} sense.

To summarise, Hare states that the purpose of moral words such as ``good'' and ``ought'' is to answer questions about how we should behave, so in accepting moral judgements, we commit ourselves to a particular standard about how to act in principle under particular circumstances. Based on examples like Jørgensen's paradox, imperatives are subject to their own logic. We therefore have a treatment of moral judgements in which they are logical, objective, universal, and prescriptive.

\section{Are Universal Imperatives Coherent?}

Moral judgements entail a universal rule about how to act in principle under particular circumstances. This is sometimes called a ``universal imperative''. John Ibberson challenges whether this is a coherent idea\footnote{~Ibberson, 1979: 156.}. His argument is as follows. The function of an imperative is to tell someone to make some indicative the case. For example, ``turn the light on'' is an instruction for someone to make true the indicative ``the light is on''. It is only sensible to instruct someone to do something now, or in the future, but never in the past; a person cannot make ``the light is on'' true in the past. And this, Ibberson says, is meaningless: ``It is hardly reasonable to tell people to make it the case that they and everyone else have always in the past spoken the truth... Yet if Hare is right this is exactly what we are doing in asserting a moral judgement''\footnote{~ibid: 158.}.

Ibberson's doubts seem to be based on a misunderstanding of Hare's theory. Hare is very clear that the relation between moral judgements and imperatives is not one of equivalence, but entailment: ``...value-judgements, if they are action-guiding, must be held to entail imperatives...''\footnote{~ibid: 163.}. Furthermore, the imperative entailed is not universal in the sense that it compels us to make some indicative true at every point in time, but rather in that it applies to every person acting on principle in like circumstances. To illustrate the difference, consider the prescription ``if it's dark you ought to turn the lights on''. If you assent to this, and it's not dark, the rule is followed whether you do or don't turn the lights on. It is also valid to apply prescriptions to past events. In promulgating standards of good-health, I might tell my friend ``if you want to be healthy, don't smoke''. By doing this, I am not telling them to make it the case that they never smoked. It may be that my friend smoked all their life, perhaps growing up in a time when the health risks of smoking weren't known, and so did not know the appropriate standards to follow. It would not be fair to criticise my friend for not following the standard if they were not aware of it, but regardless of what they knew at the time, it is still meaningful to evaluate their conduct in terms of the standard.

With the understanding that moral judgements are not universal imperatives, but rather only prescribe universal rules that invite us to act in the appropriate circumstances, Ibberson's doubts no longer stick.

\section{Problems With Imperative Logic}

Hare's theory relies on imperatives being subject to their own logic, but he does not say how that would look. It is not necessary to have a completely formal treatment of imperatives, but a tighter understanding of how they are to be interpreted is needed so we can be sure the idea is coherent, and to more precisely locate their role in moral judgements.

The first challenge to an imperative logic is one that Hare himself briefly mentions: whether there is a meaningful distinction between indicative and imperative statements. It could be that imperatives reduce to indicatives, like how Carnap reduced moral judgements to imperatives. This would be a serious blow to Hare's theory, as there would be no way to derive a commitment to act --- an \textit{ought} --- from purely indicative premises --- an \textit{is}. Hare spends a considerable amount of time describing the is-ought problem and related fallacies, and why ethical theories which are naturalistic, descriptivist, or indicative cannot avoid them\footnote{Hare, 1952: 38-55.}. If universal prescriptivism is to be a coherent theory, it therefore requires a rigid distinction between imperative and indicative logics.

Hare is adamant that imperatives cannot be reduced to indicatives, have no truth-values, and cannot be subject to the same logics as indicatives. Confusing the matter is his claim that imperatives obey a kind of non-contradiction, as when the two imperatives ``turn the lights on'' and ``turn the lights off'' are issued\footnote{~ibid: 24.}. This seems plausible at first, but a closer analysis might suggest otherwise. What about this example makes us say the two imperatives contradict one another? Perhaps it's because they imagine two futures, one where ``the lights are on'' and another where ``the lights are off''. The two futures cannot both be realised at once. But if the future conveyed is what determines whether imperatives are contradictory, then for all logical purposes we can treat them as temporal statements like ``the lights will be on'' or ``the lights will be off'' --- and this logic is indicative.

We might also say the two imperatives are contradictory because they signal conflicting desires by the speaker, but this interpretation would again reduce them to psychological or preferential indicatives like ``I want the lights to be off''. Hare even explicitly rejects this interpretation\footnote{~ibid: 6, 12.}.

The two imperatives might be called contradictory because a listener cannot do both of them. We must be careful here, because in some sense a listener can do both: by first turning the lights on, and then turning them off. If we say that a speaker cannot do both \textit{at once}, this seems to be trivially true for any pair of imperatives, because a person can only do one thing at a time: you can ``turn the lights on'', or you can ``cook dinner'', but you can't do both of these \textit{at the same time}. While it is indeed (usually) contradictory for the listener to assent to doing two commands at once, the contradiction really has nothing to do with the imperatives being issued, or the information inside them. Lastly, if what's important is whether it is \textit{possible} to realise an imperative, we still seem to be understanding imperatives logically by turning them into indicatives: ``turn the lights on'' means ``it is possible for you to turn the lights on'', which is a reduction of imperative logic to some form of modal indicative logic.

Put another way, what is it that makes an inference in imperative logic ``valid''? In indicative logics, validity is a relation between the truth-conditions of the premises and the truth-condition of the conclusion. But imperatives don't have truth-conditions, or even truth-values. We can understand what it would mean for indicatives to be true, valid, or relevant, perhaps by imagining a world in which the sense of the statement is true. But what does the world look like when an imperative statement is true, valid, or relevant? Perhaps we are misguided, and rather than having truth and validity, imperatives have ``truth'' and ``validity''. We are left wanting an explanation of ``truth'' and ``validity'', how they differ from the regular notions, and how they constitute a logic. Without at least a rough understanding of the logic of imperatives at least looks like, the claim that moral judgements and the imperatives they entail are logical entities is put on shaky grounds.

\section{The Senses of an Imperative}

We have already examined a few possible ways to interpret when an imperative might be considered ``true'' or ``valid''. None of those accounts could distinguish indicative and imperative logics, suggesting there might not be a difference. Hare seemed to favour an interpretation of imperatives based on the obligations carried by the listener who assents to them. He cites von Wright's deontic logic for an insight into what this looks like\footnote{~Hare, 1953: 27.}. But von Wright's logic does not seem to make a rigid distinction between indicative and imperative statements. His deontic logic ``studies propositions about the obligatory, permitted, forbidden, and other deontic characters of acts.''\footnote{~von Wright, 1951: 5.} And such propositions are truth-valued indicatives, like ``I am obligated to give to charity''. This leaves us in the same situation, where imperatives can only be logically scrutinised when they are interpreted in some indicative sense.

To shed light on the situation, let us examine some what senses an imperative conveys. Firstly, by issuing a command, we invite the listener to imagine a possible future, identified by those indicatives of it which are true or false. For example, ``turn the lights on'' invites us to imagine a future in which the statement ``the lights are on'' is true. Secondly, the speaker desires in some way --- be that moral, pragmatic, preferential, or legal --- that the imagined future is realised. So ``turn the lights on'' conveys ``I want it to become true that the lights are on''. The third sense conveyed is that, if the speaker assents to an imperative, there is now an obligation for them to carry out the action. A genuine imperative therefore seems to convey at least three indicative senses:\\

(1) A (theoretical) state of affairs to be realised.\\
\indent
(2) A desire by the speaker to see the state of affairs realised.\\
\indent
(3) An obligation to realise the state of affairs, if they assent to the command.
\\

By analysing an imperative in terms of these senses, we seem to get a pretty good understanding of what it means to issue and realise it. Let's take a tricky example by Hare, ``tell your father I called''\footnote{~Hare, 1953: 7.}. Ignoring any possible evaluative meaning and treating it as a pure imperative, the sentence conveys at least these descriptive meanings:\\

(1) Your father will know that I called.\\
\indent
(2) I want you to let your father know that I called.\\
\indent
(3) By assenting, you are obligated to ensure your father knows I called.\\

If there can be imperative inferences, an interpretation needs to be given which says when an inference can be applied, and what it means. Since moral judgements entail imperatives, we need to know the sense in which an imperative argument describes or compels, so we know the sense in which a moral argument describes or compels. With the above senses, we can interpret Jørgensen's paradox in at least three ways. Recall the paradox:\\

(1) Keep your promises.\\
\indent
(2) This is a promise of yours.\\
\indent
$\therefore$ Keep this promise.\\

In the first sense, an imperative sentence imagines a future to be made true by describing what it looks like. That is, we are asked to ensure some indicative is true in the future. In this sense, an imperative inference concludes a new statement of temporal logic that will be true whenever the statements of temporal logic in the premises are true. Under this interpretation, it would be valid to issue the imperative associated with the temporal statement in the conclusion, because its future is compatible with those in the premises.\\

(1) All your promises will be kept.\\
\indent
(2) This is a promise of yours.\\
\indent
$\therefore$ This promise will be kept.\\

In the second sense, the inference concludes a new desire that is logically consistent for the speaker to hold. For example, if I wanted you to keep your promises, I could also want you to keep a specific promise without my desires being in conflict.\\

(1) I want you to keep your promises.\\
\indent
(2) This is a promise of yours.\\
\indent
$\therefore$ I want you to keep this promise.\\

In the third sense, the inference says that by assenting to the commands in the premises, we also assent to the command in the conclusion; if we are obligated to keep all our promises, then we are obligated to keep this particular promise.\\

(1) I am obligated to keep my promises.\\
\indent
(2) This is a promise of mine.\\
\indent
$\therefore$ I am obligated to keep this promise.\\

Note that for a given imperative, the three interpretations are not always applicable at the same time: imagine our house is burning and I say ``save the dog'', but the fire is too dangerous or the dog is already dead, so there is no way to carry out that action. Then in the first sense, an inference using ``save the dog'' as a premise would not apply in any situation --- since the dog is necessarily dead in all futures --- while in the second and third sense, you could assent to the command (not knowing the dog is dead) and I could want you to realise it (because \textit{I} don't know the dog is dead, or I want to endanger your life). Since the corresponding inferences in the interpretations can't always be applied at the same time, we should choose the one which is most relevant for the purpose of moral language as Hare has outlined. This is the third interpretation, where imperative logic derives new obligations from previous obligations. This is also much like the deontic logic considered by von Wright and endorsed by Hare. Therefore, moving forward, regardless of whether imperative statements and inferences are reducible to indicatives or not, they should be interpreted as being about our obligations to act.

\section{Moral Knowledge}

We have seen that, by assenting to a moral judgement, we commit ourselves to accepting a sort of imperative which is universally applicable in that all rationally-behaving moral agents who assent to the judgement are obligated to act in the manner prescribed under like circumstances. We now shift our attention to moral knowledge: what does it mean when a moral judgement is true, and how do we know when they are true?

The first issue relates to the interpretation of imperative logic. As we established in the previous section, regardless of whether we do or do not accept a distinction between indicative and imperative logics, the relevant factor in the validity of an imperative statement or inference is how our obligations relate to each other. But this interpretation is problematic for a theory of ethics, because obligations are not inherent. Rather, they are acquired, accepted, and rejected. If you say ``do X'', and I do not assent, I am not obligated to do X, yet I am still acting in a rational manner. If we have previous obligations and derive new ones with the rules of imperative logic, a rational agent must assent to these new obligations. But from where are the original commitments obtained? Put another way, although we can accept a commitment to act towards good ends, evaluate something as good, and then derive an additional commitment to act on that something, it is not clear how we should understand the value-word ``good'' as being something objective that transcends personal taste or preference. This is a problem, because to act morally on the view of universal prescriptivism is to act in a logically consistent manner with respect to moral language. But in the interpretation of imperative logic implied by Hare's work --- that of an obligation to act in a certain manner --- we are only obligated to perform an imperative if we assent to it, and nothing requires us to do that.

We are thus required to treat moral judgements as relative. They may not be relative to the individual, but they are at least relative to moral communities; we could might call a person irrational for failing to act on obligations derived from the principles of their moral community, but this would carry no objective or normative weight outside of that community. The ``universal'' aspect of prescriptivism seems only to quantify over the members and circumstances of a social group, and so prescriptivism, while still a useful account of how we use language to establish and assimilate moral facts, is no longer a normative theory of ethics or morality.

Hare firmly rejects the claim that his prescriptivism is relativistic. Take the example of the two prescriptions, ``it was wrong to take the money'' and ``it was right to take the money''. Hare agrees that one of the prescriptions must be wrong, because the sentence ``one of the prescriptions must be wrong'' can be expanded to ``either it is wrong to say that it was wrong to take the money, or it is wrong to say that it was right to take the money'', which is a tautology\footnote{~Hare, 1989: 25.}. But Hare's expansion is not faithful; the speech-acts are not merely saying whether it is wrong to take the money, but prescribing that we should or should not take money in appropriate circumstances. He has turned ``it is wrong that it was wrong to take the money'' into ``it was wrong that it was wrong to say to take the money,'' and the two are quite different. Furthermore, the two imperatives need not be said by the same person. The two prescriptions could still not co-exist if the speakers came from one logically consistent community, but what if the speakers are from different moral communities? One community might promulgate standards where all people are equal, while another Orwellian community might promulgate standards where some are more equal than others. So long as these two communities are never in moral dialogue with each other, there is no contradiction in each acting differently, and we can only call both true, despite their advocation of mutually contradictory principles. All this understood, it is hard to imagine how any normative system obtained from universal prescriptivism is not relative to at least moral communities.

In some ways, Hare's later work accepts this criticism, realigning prescriptivism as a strictly meta-ethical thesis on the basis that ``...moral language is a human institution. It is the business of the moral philosopher to say, not what the logical behaviour of moral terms would be like, if they were devised by and for the use of angels, but what it actually is like.''\footnote{~Hare, 1963: 73.} This entails a normative ethics which is essentially preference utilitarianism\footnote{~ibid: 118}. To be clear, Hare's position does not say that utilitarianism is a fundamentally correct, transcendent ethical principle on which we should build all moral thought. Rather, utilitarianism is a logically coherent system that follows from the treatment of moral judgements as prescribing universal rules. And while universal prescriptivism, as a meta-ethical theory, is objective, the normative systems it entails are relativised. Accepting that universal prescriptivism is a purely meta-ethical position helps illuminate the scope and purpose of Hare's work, and gives us a better understanding of its practical implications, but it also seems to be an admission that any normative system it gives us is not universal or object, and so the claims it makes are much more modest.

\section{Moral Intellectualism}

A second issue with the prescriptivist theory is \textit{akrasia}, or moral weakness. For a moral judgement to be prescriptive, it must entail a commitment for the speaker to act according to the universal rule being advocated. But it seems to be the case that people can assent to moral standards while not acting on them.

Hare barely touches upon this issue in \textit{The Language of Morals}, but he does give one indirect reason why we fail to act on moral rules. Hare establishes that sentences have both descriptive and evaluative meanings\footnote{~Hare, 1952: 111.}. Descriptive meaning is found in the indicative information a sentence conveys, while the evaluative meaning is found in its prescriptions and moral content. Even value-judgements can have descriptive meaning. For instance, in saying ``John is a good bloke'', I could be advocating the standard that we act more like John. Or, I could simply be communicating some facts about John's disposition, such as his being kind, selfless, generous with money, etc. And while these are, to some extent, evaluative judgements (what is it that makes a person kind?), they are primarily descriptive terms. Therefore the word ``good'' in ``John is a good bloke'', despite being a moral word, is not serving a moral, prescriptive purpose in this speech act. Whether we give primacy to the evaluative or the descriptive meanings of a sentence is a matter of convention, so ``good'' is a so-called \textit{Janus word}, pulling double-duty.

Now, a situation may arise in which, having always accepted certain social and moral conventions, we forget the evaluative sense of a word and treat it as purely descriptive. For example, living in a society in which we constantly refer to people as ``good blokes'', we forget the moral force of the word ``good'' when we hear it used that way. At the same time, we may unquestioningly accept that to be good is to be generous, conflate the two concepts, and say ``John is a good bloke'' to convey ``John is generous with money''. In such contexts, ``good'' is not functioning as a moral word, so everything Hare says about it prescribing behaviour, advocating standards, etc. does not apply, just as it does not apply in other descriptive contexts. This explains why I can say and believe that ``John is a good bloke'' while living a life of rum, sodomy, and the lash.

In other places, Hare suggests that in many cases of akrasia the assenting is not \textit{genuine}. For instance, he says it is tautological that we cannot sincerely assent to a moral judgement's universal rule and ``...at the same time, not perform it.... not believe it.''\footnote{~Hare, 1952: 20}. Sometimes, our decision about what to do is not made on pragmatic, not moral grounds, because of the complexities inherent in moral reasoning: ``...making up my mind what I ought to do is a much more difficult and complex matter than making up my mind what I want to do; and it is these complexities that lead to the problem of moral weakness...''\footnote{~Hare, 1961: 72.}. Hare therefore seems to hand-wave all instances of akrasia away by saying that they aren't \textit{genuine} instances where the actor assented to a moral principle. It may seem unfair for him to do this, but Hare makes his position more clear in the following passage:

\begin{displayquote}
``I performed what some have thought an evasive manoeuvre by defining `value-judgement' [and prescription] in such a way that if a man did not do what he thought he ought, he could not be using the word evaluatively.... The purpose of both manoeuvres was, however, not to evade objections but to clarify the problem by locating it where it can be seen. The problem exists for evaluative or prescriptive uses of moral words; and it is therefore necessary to know which these are. Therefore we have to exclude from the category `evaluative' or `universally prescriptive' such uses as do not belong to it.''\footnote{~ibid: 84.}
\end{displayquote}

Hare's response is genuine and tenable, but it also entails a stronger commitment he doesn't seem to be willing to enunciate: moral intellectualism. As we have already seen, the only sensible interpretations of an imperative logic are to treat imperative inferences as entailing an obligation to act on the conclusion when we are obligated to act on the premises. This requires us to, at some point, assent to certain obligations, and the logical point at which this happens is when we are assimilated into a moral community. To commit evil is therefore to be ignorant of these fundamental principles and basic obligations forming the basis of our moral community, or the standards derived from them. Universal prescriptivism seems like it must necessarily accept moral intellectualism, and this quite nicely eliminates the problem of akrasia.

\section{Conclusion}

Universal prescriptivism is an ethical theory, according to which a moral judgement like ``you ought to do X'' is a prescription for how to behave, in principle, under particular circumstances. Moral judgements are not imperatives, as Ibberson mistakenly suggested, but rather evaluative sentences which entail imperatives. Hare derives these conclusions by appealing to the way in which moral words are used and showing that moral judgements and imperatives are subject to rational inquiry and logical scrutiny. There are many deep problems with its use as an ethical theory. Firstly, Hare's theory does not elaborate on what an imperative logic looks like, and any attempt to give an interpretation appears to lead to a collapse in the distinction between imperative and indicative logics, undermining his essential descriptivist-prescriptivist dichotomy. Secondly, it cannot condemn two disjoint moral communities who advocate conflicting standards. This is clearly seen when Hare advocates a sort of preference utilitarianism for resolving moral disputes, which exposes his theory to the usual accusations of relativism. Lastly, Hare inadequately addresses the problem of \textit{akrasia} --- situations where a person assents to some moral principle, but fails to act on it. However, there are good reasons to accept a kind of moral intellectualism, and doing so resolves the problem.

If we look at Hare's prescriptivism as a purely meta-ethical thesis, it becomes a compelling account of how moral communities rationally develop and assimilate moral knowledge and language. But this means any normative systems derived from it are fundamentally relativist, and so its claims are much more modest. Regardless, as a linguistic account, it withstands many objections and performs a great service to our understanding of ethics.\\

\noindent
\textbf{Word Count: 5229}


\section*{Bibliography}

\noindent
Carnap, R. (1937). \textit{Philosophy and Logical Syntax}. Retrieved from\\ http://intersci.ss.uci.edu/wiki/eBooks/BOOKS/Carnap/Rudolf\%20Carnap.pdf\\ (\today).\\ 

\noindent
Hare, R. M. (1952). \textit{The Language of Morals}. Published by Oxford University Press.\\

\noindent
Hare, R.M. (1963). \textit{Freedom and Reason}. Published by Clarendon Press, Oxford.\\

\noindent
Hare, R.M. (1989). `Some Confusions About Subjectivity', in \textit{Essays in Ethical Theory}: 14-32. Published by Clarendon Press, Oxford.\\

\noindent
Ibberson, J. (1979). `A Doubt about Universal Prescriptivism', in \textit{Analysis} 39, 3: 153-158.\\

\noindent
Jørgensen, J. (1938). 'Imperatives and Logic', in \textit{Erkenntnis}, 7: 288-98.


\end{document}











